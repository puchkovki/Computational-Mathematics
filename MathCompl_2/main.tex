\documentclass[a4paper, 12pt]{article}
\usepackage[utf8]{inputenc}
\usepackage[T2A]{fontenc}
\usepackage[english, russian]{babel}
\usepackage{amsmath, amsfonts, amssymb, amsthm, mathtools, forest}
\usepackage{graphicx}
\usepackage{wrapfig}
\usepackage{multirow}
\usepackage{forest}
\usepackage{float}
\usepackage[a4paper, left=20mm, right=10mm, top=20mm, bottom=20mm]{geometry}

\title{Comp Math 2}
\author{puchkovki}

\begin{document}

\maketitle

\section*{Погрешности}

Оценим теоретическое влияние погрешности коэффициентов матрицы 2x2 на детерминант матрицы. Пусть a, b, c, d - коэффициенты данной матрицы, а $\gamma$ — их погрешность. Тогда: \\
\\
$\begin{vmatrix}
  a+\gamma& b+\gamma\\
  c+\gamma& d+\gamma
\end{vmatrix} -\begin{vmatrix}
  a& b\\
  c& d
\end{vmatrix}  = ad+ (a+d)\gamma + \gamma^2 - cb - (b+c)\gamma - \gamma^2 - ad + cb = (a + d - c - b)\gamma $

\section*{Оценка погрешности при переходе между сетками с шагами $2\cdot10^{-4}$ и $2\cdot10^{-5}$ соответственно в точке $(-0.5, -0.5)$}

Для оценки ошибки на нескольких сетках использовать экстраполяцию Ричардсона. Основная идея: если производная $f_h'(x)$ вычисляется по формуле с порядком аппроксимации $p$, то справедлива формула:

\[f_h'(x) = f'(x) + C_1\cdot h^p\]

с некоторой величиной $C_1$. При вычисленни производной с шагом $\frac{h}{10}$ можно записать аналогичное соотношение

\[f_{\frac{h}{10}}'(x) = f'(x) + C_2\cdot \left(\frac{h}{10}\right)^p\]
с некоторой величиной $C_2$. Пологая $C_1=C_2=C$ и решая систему, получаем:
\[C\cdot h^p = \frac{f_h'(x)-f_{\frac{h}{10}}'(x)}{1-10^{-p}},\]

откуда получаем абсолютную погрешность. 
\[f''_{xx_{h}} = \frac{f(x+h, y) + f(x-h, y) - 2\cdot f(x,y)}{4\cdot h^2} =\]

\[\frac{f(x, y) + h\cdot f'_x+\frac{h^2}{2}\cdot f_{xx}''+ C_1\cdot h^3 + f(x, y) - h\cdot f'_x+\frac{h^2}{2}\cdot f_{xx}''+ C_2\cdot h^3 - 2\cdot f(x,y)}{4\cdot h^2} = \]

\[f_{xx}'' + C\cdot h\]

Следовательно, порядок аппроксимации $p$ равен 1. Аналогично для $f_{yy}''$.

\[f''_{xy_{h}} = \frac{f(x+h, y+h) - f(x+h,y-h) + f(x-h, y-h) - f(x-h,y+h)}{4\cdot h^2} =\]

\[\frac{f(x, y) + h\cdot f'_x  + h\cdot f'_y + \frac{h^2}{2}\cdot f_{xx}''+ \frac{h^2}{2}\cdot f_{yy}'' + h^2\cdot f_{xy}'' + C_1\cdot h^3}{4\cdot h^2} - \]

\[ - \frac{(f(x, y) + h\cdot f'_x  - h\cdot f'_y + \frac{h^2}{2}\cdot f_{xx}'' + \frac{h^2}{2}\cdot f_{yy}'' - h^2\cdot f_{xy}'' + C_2\cdot h^3)}{4\cdot h^2} + \]

\[ + \frac{f(x, y) - h\cdot f'_x  - h\cdot f'_y + \frac{h^2}{2}\cdot f_{xx}'' + \frac{h^2}{2}\cdot f_{yy}'' + h^2\cdot f_{xy}'' + C_3\cdot h^3)}{4\cdot h^2} - \]

\[ - \frac{(f(x, y) - h\cdot f'_x  + h\cdot f'_y + \frac{h^2}{2}\cdot f_{xx}'' + \frac{h^2}{2}\cdot f_{yy}'' - h^2\cdot f_{xy}'' + C_4\cdot h^3)}{4\cdot h^2} = f_{xy}'' + C\cdot h\]

\begin{itemize}
    \item $xx: C\cdot h = \frac{10}{9} \cdot (0.2203932716 - 0.2203932696) = \frac{10}{9} \cdot 2\cdot 10^{-9} = \frac{20}{9}  \cdot 10^{-9} = \gamma$
    \item $yy: C\cdot h = \frac{10}{9} \cdot (0.2203932716 - 0.2203932696) = \frac{10}{9} \cdot 2\cdot 10^{-9} = \frac{20}{9}  \cdot 10^{-9} = \gamma$
    \item $xy: C\cdot h = \frac{10}{9} \cdot (1.537096842 - 1.53709686) = \frac{10}{9} \cdot -18\cdot 10^{-9} = -2 \cdot 10^{-8} = \alpha$
\end{itemize}

$\begin{vmatrix}
  a+\gamma& c+\alpha\\
  c+\alpha& b+\gamma
\end{vmatrix} -\begin{vmatrix}
  a& c\\
  c& d
    \end{vmatrix}  = (a + b)\cdot\gamma - 2\cdot\alpha\cdot c + \gamma^2-\alpha^2 = (0.2203932696 +0.2203932696)\cdot \frac{20}{9}\cdot 10^{-9} - 2\cdot 1,53709686 \cdot (-2)\cdot 10^{-8} + (\frac{20}{9}\cdot 10^{-9})^2 - ((-2)\cdot 10^{-8})^2 = 0.98\cdot 10^{-9} + 61.48\cdot 10^{-9} = 62,5\cdot 10^{-9}$
\end{document}
